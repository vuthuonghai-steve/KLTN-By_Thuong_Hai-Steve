% =============================================================================
% CHƯƠNG 2: PHÂN TÍCH VÀ THIẾT KẾ HỆ THỐNG AGENT SKILLS
% =============================================================================

\chapter{PHÂN TÍCH VÀ THIẾT KẾ HỆ THỐNG AGENT SKILLS}

Trong chương này, báo cáo trình bày chi tiết về kiến trúc tổng thể của hệ thống Agent Skills,
cũng như phân tích từng bộ kỹ năng chuyên biệt được thiết kế để tự động hóa quá trình phân tích
và thiết kế hệ thống trong giai đoạn Life-2.

% =============================================================================
\section{Kiến trúc tổng thể hệ thống Agent Skills}

Hệ thống Agent Skills được xây dựng dựa trên mô hình \textbf{Meta-Skill Framework},
bao gồm ba trụ cột chính (3 Pillars) và bảy vùng chức năng (7 Zones). Kiến trúc này
đảm bảo tính nhất quán, khả năng kiểm soát chất lượng và khả năng mở rộng của các
bộ Agent Skill.

% -----------------------------------------------------------------------------
\subsection{Mô hình Meta-Skill Framework}

Meta-Skill Framework định nghĩa cách tiếp cận tổng thể để xây dựng một Agent Skill
có cấu trúc, logic rõ ràng và khả năng tự kiểm soát chất lượng. Mô hình này bao gồm
ba giai đoạn chính:

\begin{enumerate}
    \item \textbf{Pha 1 — Nhận diện (Input/Output)}: Xác định rõ ràng đầu vào và đầu ra
    mong muốn của skill.

    \item \textbf{Pha 2 — Chiến lược (Checklist/Plan)}: Lập kế hoạch thực thi với
    checklist kiểm soát chất lượng.

    \item \textbf{Pha 3 — Thực thi (Research/Action)}: Nghiên cứu và thực hiện các
    bước được định nghĩa.
\end{enumerate}

Ba giai đoạn này được kết nối với vòng lặp kiểm soát (Verify/Fix/Loop) để đảm bảo
chất lượng đầu ra. Mô hình được minh họa trong Hình~\ref{fig:meta-skill-framework}.

\begin{figure}[h]
    \centering
    % TODO: Thêm hình ảnh sau khi export từ Mermaid
    % \includegraphics[width=0.8\textwidth]{figures/meta-skill-framework}
    \fbox{\parbox{0.8\textwidth}{\centering [Hình minh họa Meta-Skill Framework sẽ được chèn ở đây]}}
    \caption{Mô hình Meta-Skill Framework}
    \label{fig:meta-skill-framework}
\end{figure}

% -----------------------------------------------------------------------------
\subsection{Ba trụ cột (3 Pillars)}

Kiến trúc Agent Skill được xây dựng trên ba trụ cột chính, giống như một ngôi nhà
vững chắc:

\subsubsection{Pillar 1 — Knowledge (Tri thức)}

Tập hợp các quy định, tiêu chuẩn kỹ thuật (UML, Schema Design Patterns, Best Practices)
cung cấp "Context" cho AI Agent. Tri thức được tổ chức trong thư mục \texttt{knowledge/}
và được nạp theo chiến lược Progressive Disclosure (nạp dần theo nhu cầu).

Ví dụ tri thức trong một skill:
\begin{itemize}
    \item \texttt{knowledge/uml-rules.md}: Quy tắc vẽ UML diagram chuẩn
    \item \texttt{knowledge/mongodb-patterns.md}: Các mẫu thiết kế MongoDB
    \item \texttt{knowledge/payload-types.md}: Các loại field trong PayloadCMS
\end{itemize}

\subsubsection{Pillar 2 — Process (Quy trình)}

Các bước thực thi được module hóa thành workflow rõ ràng, từ nhận diện đầu vào đến
kiểm chứng đầu ra. Mỗi skill định nghĩa riêng quy trình phù hợp với domain của nó.

Ví dụ quy trình 5-phase workflow:
\begin{enumerate}
    \item \textbf{Phase 1 — Collect}: Thu thập thông tin đầu vào
    \item \textbf{Phase 2 — Research}: Nghiên cứu codebase và tài liệu
    \item \textbf{Phase 3 — Design}: Thiết kế giải pháp
    \item \textbf{Phase 4 — Generate}: Sinh output (diagram, code, spec)
    \item \textbf{Phase 5 — Verify}: Kiểm chứng chất lượng
\end{enumerate}

\subsubsection{Pillar 3 — Guardrails (Kiểm soát)}

Các hàng rào bảo vệ chống lại hiện tượng "hallucination" (ảo giác) của AI thông qua
các cơ chế:

\begin{itemize}
    \item \textbf{Interaction Gates}: Các điểm dừng bắt buộc để hỏi user xác nhận
    \item \textbf{Source Citation}: Bắt buộc mọi field/entity phải có nguồn trích dẫn
    \item \textbf{Self-Scoring}: AI tự đánh giá chất lượng output theo rubric định trước
    \item \textbf{Checklist Verification}: Kiểm tra output với danh sách câu hỏi chuẩn
\end{itemize}

% -----------------------------------------------------------------------------
\subsection{Bảy vùng chức năng (7 Zones)}

Mỗi Agent Skill được tổ chức thành 7 zones chuẩn như mô tả trong Bảng~\ref{tab:7-zones}.
Cấu trúc này đảm bảo tính đồng nhất giữa các skills và dễ dàng bảo trì.

\begin{table}[h]
\centering
\caption{Bảng mô tả 7 Zones trong Agent Skill}
\label{tab:7-zones}
\begin{tabular}{|p{0.15\textwidth}|p{0.35\textwidth}|p{0.35\textwidth}|}
\hline
\textbf{Zone} & \textbf{Mục đích} & \textbf{Ví dụ nội dung} \\
\hline
Core & Linh hồn điều khiển — Persona, Workflow, Guardrails &
Persona: Senior Architect; 5-phase workflow; Interaction Points \\
\hline
knowledge/ & Tri thức chuẩn — Standards, Best practices &
UML rules, Design patterns, MongoDB patterns \\
\hline
scripts/ & Công cụ tự động hóa — Python, Bash, JavaScript &
analyzer.py, validator.py, generator.py \\
\hline
templates/ & Mẫu đầu ra — Code stubs, Diagram templates &
sequence.mmd, class.mmd, design.md.template \\
\hline
data/ & Cấu hình và dữ liệu cứng — Config YAML, JSON schema &
config.yaml, allowed-types.json \\
\hline
loop/ & Kiểm soát chất lượng — Checklist, Test cases &
checklist.md, phase-verify.md, test-cases/ \\
\hline
assets/ & Tài nguyên tĩnh — Icons, Fonts, Images &
icons/, fonts/, images/ \\
\hline
\end{tabular}
\end{table}

% -----------------------------------------------------------------------------
\subsection{Quy trình 5 bước xây dựng Skill}

Quy trình xây dựng một Agent Skill tuân theo 5 bước chuẩn được minh họa trong
Hình~\ref{fig:5-step-workflow}:

\begin{enumerate}
    \item \textbf{Bước 1 — Khảo sát (Research \& Discovery)}:
    \begin{itemize}
        \item Xác định Input: Người dùng sẽ đưa cái gì? (Rác hay vàng?)
        \item Xác định Tools: AI sẽ dùng Terminal, Browser hay Library nào?
        \item Tìm ra "Điểm mù" của AI: AI thường sai ở đâu trong công việc này?
    \end{itemize}

    \item \textbf{Bước 2 — Thiết kế (System Design)}:
    \begin{itemize}
        \item Xây dựng Quy trình logic (Flowchart): Bước A → Bước B
        \item Xác định Điểm dừng tương tác: Khi nào AI PHẢI hỏi người dùng?
        \item Định nghĩa Định dạng Output: Mermaid, Markdown, hay Code?
    \end{itemize}

    \item \textbf{Bước 3 — Xây dựng (Build)}:
    \begin{itemize}
        \item Persona: Định vị AI là Senior Architect hay Senior Coder
        \item Phase-based Steps: Chia nhỏ công việc thành các Pha
        \item Tạo templates, scripts, knowledge files
    \end{itemize}

    \item \textbf{Bước 4 — Kiểm định (Verify)}:
    \begin{itemize}
        \item Chạy Test Cases
        \item Verify Checklist
        \item Rollback nếu phát hiện lỗi
    \end{itemize}

    \item \textbf{Bước 5 — Bảo trì (Maintenance)}:
    \begin{itemize}
        \item Feedback Loop: Ghi lại chỗ AI làm dở
        \item Version Control: Cập nhật khi môi trường thay đổi
    \end{itemize}
\end{enumerate}

\begin{figure}[h]
    \centering
    % TODO: Thêm hình ảnh sau khi export từ Mermaid
    % \includegraphics[width=0.9\textwidth]{figures/5-step-workflow}
    \fbox{\parbox{0.9\textwidth}{\centering [Hình minh họa 5-step workflow sẽ được chèn ở đây]}}
    \caption{Quy trình 5 bước xây dựng Agent Skill}
    \label{fig:5-step-workflow}
\end{figure}

% =============================================================================
\section{Bộ kỹ năng kiến trúc sư (Skill Architect)}

Skill Architect là meta-skill trung tâm, chịu trách nhiệm thiết kế cấu trúc cho các
Agent Skills khác. Đây là điểm khởi đầu của toàn bộ Skill Suite trong pipeline
\texttt{Architect → Planner → Builder}.

% -----------------------------------------------------------------------------
\subsection{Vai trò và vị trí trong pipeline}

Skill Architect đóng vai trò như một "kiến trúc sư trưởng", nhận yêu cầu từ người dùng
và tạo ra bản thiết kế chi tiết (\texttt{design.md}) cho skill mới. Bản thiết kế này
tuân theo format chuẩn 10 sections:

\begin{enumerate}
    \item Problem Statement — Mô tả vấn đề cần giải quyết
    \item Capability Map — Phân tích 3 Pillars (Knowledge, Process, Guardrails)
    \item Zone Mapping — Contract bắt buộc giữa Architect và Planner
    \item Folder Structure — Sơ đồ Mermaid mindmap
    \item Execution Flow — Các sơ đồ sequence/flowchart minh họa
    \item Interaction Points — Các điểm dừng tương tác với user
    \item Progressive Disclosure Plan — Chiến lược nạp tài nguyên (Tier 1/2)
    \item Risks \& Blind Spots — Rủi ro và cách mitigation
    \item Open Questions — Câu hỏi còn mở cần giải đáp
    \item Metadata — Thông tin phiên bản, tác giả, trạng thái
\end{enumerate}

Vị trí của Skill Architect trong pipeline được minh họa trong Hình~\ref{fig:skill-suite-pipeline}.

\begin{figure}[h]
    \centering
    % TODO: Thêm hình ảnh pipeline
    % \includegraphics[width=0.9\textwidth]{figures/skill-suite-pipeline}
    \fbox{\parbox{0.9\textwidth}{\centering [Sơ đồ Pipeline: Architect → Planner → Builder → Skill Files]}}
    \caption{Pipeline Skill Suite: Architect → Planner → Builder}
    \label{fig:skill-suite-pipeline}
\end{figure}

% -----------------------------------------------------------------------------
\subsection{Quy trình Adaptive Workflow}

Skill Architect sử dụng quy trình Adaptive Workflow, tự động phân loại độ phức tạp
của yêu cầu và chọn workflow phù hợp:

\begin{itemize}
    \item \textbf{Simple}: DETECT → COLLECT (rút gọn) → DESIGN (merge Analyze + Design) → DONE
    \item \textbf{Medium}: DETECT → COLLECT → ANALYZE → DESIGN → DONE
    \item \textbf{Complex}: DETECT → COLLECT → ANALYZE → ARCH-REVIEW → DESIGN → DONE
\end{itemize}

Trong đó:
\begin{itemize}
    \item \textbf{DETECT}: Đọc input, áp \texttt{complexity-matrix.md}, chọn path tự động
    \item \textbf{COLLECT}: Hỏi đúng gap còn thiếu (không hỏi lại những gì đã biết)
    \item \textbf{ANALYZE}: Phân tích 3 Pillars + Zone Mapping contract (§3 bắt buộc)
    \item \textbf{ARCH-REVIEW}: (Complex only) Kiểm tra dependency với skill-planner/builder
    \item \textbf{DESIGN}: Tạo ≥3 Mermaid diagrams + self-score từng section trước deliver
\end{itemize}

% -----------------------------------------------------------------------------
\subsection{Cơ chế Self-Scoring và Quality Gates}

Skill Architect tích hợp cơ chế tự đánh giá (Self-Scoring) để đảm bảo chất lượng
thiết kế trước khi chuyển sang giai đoạn thực thi. Các tiêu chí đánh giá được
mô tả trong Bảng~\ref{tab:self-scoring}.

\begin{table}[h]
\centering
\caption{Rubric tự đánh giá Skill Architect}
\label{tab:self-scoring}
\begin{tabular}{|l|p{0.5\textwidth}|c|}
\hline
\textbf{Section} & \textbf{Tiêu chí} & \textbf{Điểm tối thiểu} \\
\hline
§1 Problem & Pain Point rõ ràng, User xác định, Output cụ thể & 3/5 \\
\hline
§2 Capability & 3 Pillars đầy đủ, Process có workflow diagram & 3/5 \\
\hline
§3 Zone Mapping & Tên file cụ thể (regex compliant), không placeholder & 4/5 \\
\hline
§4 Folder & Có Mermaid mindmap rõ ràng & 3/5 \\
\hline
§5 Execution Flow & Có ≥2 diagrams (sequence/flowchart) & 3/5 \\
\hline
§6 Interaction Points & Có ≥2 IPs với lý do dừng rõ ràng & 3/5 \\
\hline
§7 Progressive Disclosure & Phân biệt rõ Tier 1 (mandatory) và Tier 2 (conditional) & 3/5 \\
\hline
§8 Risks & Có ≥5 risks kèm mitigation cụ thể & 3/5 \\
\hline
§9 Open Questions & Có ≥2 câu hỏi thực chất (không phải placeholder) & 3/5 \\
\hline
§10 Metadata & Đầy đủ thông tin: name, version, status, handoff checklist & 3/5 \\
\hline
\end{tabular}
\end{table}

Quy tắc: Nếu bất kỳ section nào có điểm < 3/5, AI phải re-work section đó trước khi deliver.

% -----------------------------------------------------------------------------
\subsection{Zone Mapping Contract}

§3 Zone Mapping trong \texttt{design.md} là contract bắt buộc giữa Architect và Planner.
Nó quy định rõ ràng các file cần tạo, nội dung từng zone, và trạng thái bắt buộc/tùy chọn.

Ví dụ Zone Mapping cho Skill Architect được mô tả trong Bảng~\ref{tab:zone-mapping-architect}.

\begin{table}[h]
\centering
\caption{Zone Mapping Contract của Skill Architect}
\label{tab:zone-mapping-architect}
\begin{tabular}{|l|p{0.3\textwidth}|p{0.3\textwidth}|c|}
\hline
\textbf{Zone} & \textbf{Files} & \textbf{Nội dung} & \textbf{Bắt buộc?} \\
\hline
Core & SKILL.md & Persona v2, Adaptive workflow, Guardrails & ✅ \\
\hline
Knowledge & complexity-matrix.md & Bảng phân loại Simple/Medium/Complex & ✅ \\
\hline
Knowledge & zone-contract-spec.md & Schema §3, regex validation, examples & ✅ \\
\hline
Scripts & init\_context.py & Khởi tạo .skill-context/\{name\}/ & ✅ \\
\hline
Templates & design.md.template & 10-section template cập nhật & ✅ \\
\hline
Loop & design-checklist.md & Quality gate tổng (cập nhật) & ✅ \\
\hline
Loop & phase-verify.md & Per-phase checklist & ✅ \\
\hline
\end{tabular}
\end{table}

% =============================================================================
\section{Bộ kỹ năng phân tích luồng nghiệp vụ}

Hệ thống bao gồm ba bộ kỹ năng chuyên biệt cho việc phân tích và thiết kế các
luồng nghiệp vụ: Sequence Design Analyst, Flow Design Analyst, và Activity Diagram
Design Analyst.

% -----------------------------------------------------------------------------
\subsection{Sequence Design Analyst}

\subsubsection{Vai trò}

Sequence Design Analyst chuyên trách việc thiết kế Sequence Diagram (sơ đồ tuần tự)
chuẩn UML, mô tả tương tác giữa các objects theo thời gian.

\subsubsection{Quy trình 5 pha}

\begin{enumerate}
    \item \textbf{Phase 1 — Scenario Discovery}: Phân tích kịch bản từ input và \texttt{context1.md}
    \item \textbf{Phase 2 — Codebase Research}: Quét codebase xác định classes/methods tham gia
    \item \textbf{Phase 3 — Traceability Analysis}: Xây dựng chuỗi tương tác (ai gọi ai, tham số gì)
    \item \textbf{Phase 4 — Drafting \& Refinement}: Sinh code Mermaid và tối ưu layout
    \item \textbf{Phase 5 — Quality Assurance}: Kiểm tra chéo với Checklist trước khi bàn giao
\end{enumerate}

\subsubsection{Guardrails}

\begin{itemize}
    \item \textbf{Code-First Truth}: Chỉ vẽ những gì thực sự tồn tại trong codebase
    \item \textbf{Readability Limit}: Cảnh báo nếu sơ đồ quá phức tạp (> 8 lifelines)
    \item \textbf{Naming Consistency}: Tên Actor/Object khớp 100\% với codebase
\end{itemize}

\subsubsection{Sản phẩm đầu ra}

File Mermaid sequence diagram (\texttt{.mmd}) với format chuẩn:
\begin{verbatim}
sequenceDiagram
    participant U as User
    participant S as System
    participant DB as Database

    U->>S: Login request
    S->>DB: Query user
    DB-->>S: User data
    S-->>U: Access granted
\end{verbatim}

% -----------------------------------------------------------------------------
\subsection{Flow Design Analyst}

\subsubsection{Vai trò}

Flow Design Analyst thiết kế Business Process Flow Diagram (High-Fidelity) theo
chuẩn 3-lane Swimlane (User / System / Database).

\subsubsection{Đặc điểm nổi bật}

\begin{itemize}
    \item \textbf{3-lane Swimlane}: Phân định rõ trách nhiệm xử lý dữ liệu
    \item \textbf{Auto Discovery}: Tự động phân tích intent và khám phá tài nguyên dự án
    \item \textbf{Trích xuất logic}: Từ spec/user-story → Mermaid flowchart
\end{itemize}

\subsubsection{Sản phẩm đầu ra}

Mermaid flowchart với swimlane rõ ràng, minh họa toàn bộ luồng nghiệp vụ từ
User action → System processing → Database operations.

% -----------------------------------------------------------------------------
\subsection{Activity Diagram Design Analyst}

\subsubsection{Vai trò}

Activity Diagram Design Analyst phân tích và thiết kế sơ đồ Activity Diagram
(High-Fidelity) theo tư duy Clean Architecture (B-U-E: Boundary-UseCase-Entity).

\subsubsection{Khả năng phản biện}

Skill này không chỉ vẽ diagram mà còn:
\begin{itemize}
    \item \textbf{Phát hiện Deadlocks}: Tìm các luồng có thể bị kẹt (vòng lặp vô hạn)
    \item \textbf{Kiểm tra nhất quán}: Đảm bảo logic trong diagram khớp với nghiệp vụ
    \item \textbf{Refactor Risk Detection}: Cảnh báo khi logic quá phức tạp cần refactor
\end{itemize}

\subsubsection{Sản phẩm đầu ra}

Activity Diagram với phân tầng B-U-E rõ ràng, kèm báo cáo phân tích rủi ro.

% =============================================================================
\section{Bộ kỹ năng thiết kế cấu trúc dữ liệu (Class Diagram Analyst)}

Class Diagram Analyst (Skill 2.5) đảm nhiệm việc chuyển đổi từ ER Diagram và quy trình
nghiệp vụ sang Class Diagram với định dạng dual-format (Mermaid + YAML Contract).

% -----------------------------------------------------------------------------
\subsection{Dual-format output strategy}

Skill tạo ra hai loại output cho hai đối tượng khác nhau:

\begin{itemize}
    \item \textbf{Mermaid classDiagram} (\texttt{class-mX.md}): Dạng trực quan cho
    con người review, bao gồm sơ đồ lớp và bảng Traceability (field → source mapping).

    \item \textbf{YAML Contract} (\texttt{class-mX.yaml}): Dạng machine-readable cho
    AI Agent đọc (đặc biệt là \texttt{schema-design-analyst} ở giai đoạn tiếp theo).
    File này có header LOCKED để tránh chỉnh sửa thủ công.
\end{itemize}

% -----------------------------------------------------------------------------
\subsection{Aggregate Root vs Embedded Document}

Một trong những quyết định quan trọng là phân loại entity thành Aggregate Root
(collection độc lập) hoặc Embedded Document (nhúng trong parent). Decision tree
được mô tả trong Hình~\ref{fig:aggregate-decision}.

Quy tắc quyết định:
\begin{enumerate}
    \item Nếu nhiều collection khác FK trỏ vào → \textbf{Aggregate Root}
    \item Nếu entity có timestamps riêng (createdAt, updatedAt) → \textbf{Aggregate Root}
    \item Nếu query entity độc lập (không thông qua parent) → \textbf{Aggregate Root}
    \item Nếu size có thể vượt giới hạn 16MB MongoDB → \textbf{Aggregate Root}
    \item Ngược lại → \textbf{Embedded Document}
\end{enumerate}

\begin{figure}[h]
    \centering
    % TODO: Thêm hình decision tree
    % \includegraphics[width=0.85\textwidth]{figures/aggregate-decision-tree}
    \fbox{\parbox{0.85\textwidth}{\centering [Decision Tree: Aggregate Root vs Embedded Document]}}
    \caption{Decision Tree: Aggregate Root vs Embedded Document}
    \label{fig:aggregate-decision}
\end{figure}

% -----------------------------------------------------------------------------
\subsection{Source Citation mechanism}

Để chống hallucination, mọi field trong Class Diagram PHẢI có source citation rõ ràng.
Ví dụ trong YAML Contract:

\begin{verbatim}
fields:
  - name: "email"
    type: "email"
    required: true
    unique: true
    indexed: true
    source: "er-diagram.md#L169"
  - name: "bio"
    type: "text"
    source: "activity-diagrams/m1-a1-registration.md#L45"
\end{verbatim}

Guardrail bắt buộc:
\begin{itemize}
    \item Field không có source → BLOCK, không ghi file
    \item Field có source từ file context (không có trong ER) → Mark \texttt{[FROM\_CONTEXT]}
    \item Field không tìm thấy nguồn nào → Mark \texttt{[ASSUMPTION]}, alert user ngay IP1
\end{itemize}

% -----------------------------------------------------------------------------
\subsection{Quy trình 6 phases với 3 Interaction Points}

Skill thực thi theo 6 phases với 3 Interaction Points bắt buộc:

\begin{enumerate}
    \item \textbf{Phase A — Extract Entities}: Đọc \texttt{er-diagram.md}, lấy entity list + fields
    \item \textbf{Phase B — Cross-Reference}: Grep \texttt{activity-diagrams/} tìm Hooks/Behavior
    \item \textbf{Phase C — Classify}: Quyết định Root/Embed theo decision tree
    \item \textbf{[IP1] Confirm Entity List}: CHỜ user xác nhận danh sách entities
    \item \textbf{Phase D — Generate Markdown}: Sinh \texttt{class-mX.md}
    \item \textbf{[IP2] Review Markdown}: CHỜ user approve file .md
    \item \textbf{Phase E — Generate YAML}: Chuyển .md → .yaml contract (LOCKED)
    \item \textbf{Phase F — Self-Validate}: Chạy \texttt{validate\_contract.py}
    \item \textbf{[IP3] Report Result}: Báo cáo validation pass/fail
\end{enumerate}

Interaction Points đảm bảo AI không đi sai hướng khi xử lý. Workflow được minh họa
trong Hình~\ref{fig:class-6-phase}.

\begin{figure}[h]
    \centering
    % TODO: Thêm hình workflow 6 phases
    % \includegraphics[width=0.95\textwidth]{figures/class-diagram-6-phase}
    \fbox{\parbox{0.95\textwidth}{\centering [Workflow 6 Phases với 3 Interaction Points]}}
    \caption{Workflow 6 Phases của Class Diagram Analyst}
    \label{fig:class-6-phase}
\end{figure}

% -----------------------------------------------------------------------------
\subsection{Sản phẩm đầu ra}

Mỗi module tạo ra 2 files trong thư mục tương ứng:

\begin{itemize}
    \item \texttt{Docs/life-2/diagrams/class-diagrams/m1-auth-profile/class-m1.md}
    \item \texttt{Docs/life-2/diagrams/class-diagrams/m1-auth-profile/class-m1.yaml}
\end{itemize}

File \texttt{.md} phục vụ human review, file \texttt{.yaml} là input cho
\textbf{schema-design-analyst} (Skill 2.6) ở giai đoạn tiếp theo.

% =============================================================================
\section{Bộ kỹ năng phân tích kiến trúc giao diện (UI Architecture Analyst)}

UI Architecture Analyst đóng vai trò là "cầu nối" giữa logic hệ thống và giao diện
người dùng. Skill này chuyển đổi từ Schema + Diagrams → UI Component Specs.

% -----------------------------------------------------------------------------
\subsection{Data-Component Binding}

Skill thực hiện ánh xạ (Mapping) từ các trường dữ liệu trong Schema sang các thành phần
giao diện thực tế (UI Components) dựa trên bảng mapping rules trong
\texttt{knowledge/mapping-rules.md}.

Ví dụ ánh xạ:

\begin{table}[h]
\centering
\caption{Ánh xạ Schema Type → UI Component}
\label{tab:schema-ui-mapping}
\begin{tabular}{|l|l|l|}
\hline
\textbf{Schema Type} & \textbf{UI Component (shadcn)} & \textbf{Props} \\
\hline
text & Input & type="text", placeholder, maxLength \\
\hline
email & Input & type="email", required \\
\hline
select & Select & options[], defaultValue \\
\hline
richText & RichTextEditor & toolbar config \\
\hline
upload & ImageUpload & maxSize, accept \\
\hline
relationship & RelationPicker & collection, multiple \\
\hline
\end{tabular}
\end{table}

Quy tắc:
\begin{itemize}
    \item \textbf{Zero Hallucination}: Không thêm UI field nếu không có trong Schema
    \item \textbf{Validation Sync}: Validation rules từ Schema → Component props
    \item \textbf{Required/Optional}: Khớp với Schema definition
\end{itemize}

% -----------------------------------------------------------------------------
\subsection{Screen Inventory và UI Contract}

Skill tạo ra file \texttt{mX-ui-spec.md} theo format 3-section:

\subsubsection{Section 1 — Screen Inventory}

Bảng liệt kê màn hình với format:
\begin{itemize}
    \item Screen ID: \texttt{SC-M[X]-0N}
    \item Tên màn hình
    \item Mục tiêu (User goal)
    \item Actor (ai sử dụng)
    \item Use Case reference
\end{itemize}

\subsubsection{Section 2 — Detailed Screen Logic}

Mỗi màn hình có 3 sub-sections:
\begin{itemize}
    \item \textbf{(A) Data-Component Binding Table}: UI Element → Source Field → Component Type
    \item \textbf{(B) Interaction Flow}: Pre-conditions, Post-conditions, Error handling
    \item \textbf{(C) States \& Variations}: Loading, Empty, Error states
\end{itemize}

\subsubsection{Section 3 — UI Contract}

Bảng mapping \texttt{UI ID → data-testid} cho AI Life-3 code được ngay.

% -----------------------------------------------------------------------------
\subsection{Quy trình 5 phases}

\begin{enumerate}
    \item \textbf{Phase 1 — Context Discovery}: Module ID → Danh sách files cần đọc
    \item \textbf{Phase 2 — Screen Identification}: Flow/Use Case → Screen Inventory draft
    \item \textbf{[IP-1]}: Confirm Screen Inventory với user
    \item \textbf{Phase 3 — Data \& Component Mapping}: Schema + Inventory → Binding table
    \item \textbf{Phase 4 — Synthesis \& Merge}: Merge với file cũ (nếu có)
    \item \textbf{[IP-2]}: Nếu file cũ tồn tại, confirm merge plan
    \item \textbf{Phase 5 — Output Generation}: Ghi file \texttt{mX-ui-spec.md}
\end{enumerate}

% -----------------------------------------------------------------------------
\subsection{Sản phẩm đầu ra}

File \texttt{Docs/life-2/ui/specs/m[X]-*-ui-spec.md} với đầy đủ 3 sections.
File này là input chính cho \textbf{ui-wireframe-designer} (sử dụng MCP Pencil để
vẽ wireframe trực tiếp lên canvas).

% =============================================================================
\section{Quy trình phối hợp và tích hợp}

Toàn bộ các bộ kỹ năng hoạt động phối hợp theo một pipeline tuần tự và chặt chẽ,
tạo thành một "Knowledge Factory" tự động hóa.

% -----------------------------------------------------------------------------
\subsection{Pipeline tổng thể}

Pipeline hoàn chỉnh được minh họa trong Hình~\ref{fig:full-pipeline}:

\begin{figure}[h]
    \centering
    % TODO: Thêm hình pipeline đầy đủ
    % \includegraphics[width=\textwidth]{figures/full-pipeline}
    \fbox{\parbox{\textwidth}{\centering [Full Pipeline: Life-2 Analysis \& Design]}}
    \caption{Pipeline tổng thể của hệ thống Agent Skills}
    \label{fig:full-pipeline}
\end{figure}

Các bước:
\begin{enumerate}
    \item \textbf{Thiết kế kỹ năng}: Sử dụng Skill Architect để định hình vai trò của từng Agent
    \item \textbf{Phân tích nghiệp vụ}: Flow/Sequence/Activity Analyst xây dựng khung logic
    \item \textbf{Thiết kế dữ liệu}: Class Diagram Analyst chuyển đổi sang OOP view
    \item \textbf{Thiết kế giao diện}: UI Architecture Analyst ánh xạ Schema → UI Components
    \item \textbf{Đóng gói tri thức}: Tổng hợp toàn bộ kết quả thành tài liệu kỹ thuật minh bạch
\end{enumerate}

% -----------------------------------------------------------------------------
\subsection{Interaction Points và Quality Gates}

Mỗi skill có từ 2-3 Interaction Points (IPs) bắt buộc — các điểm dừng để:
\begin{itemize}
    \item Xác nhận scope trước khi đi sâu
    \item Review draft output trước khi finalize
    \item Báo cáo validation results trước khi deliver
\end{itemize}

Quality Gates đảm bảo:
\begin{itemize}
    \item Zero hallucination (mọi field có source citation)
    \item Zero invented flow (mọi interaction có trong diagram)
    \item Consistency check (Class Diagram ↔ ER Diagram ↔ Spec)
\end{itemize}

% -----------------------------------------------------------------------------
\subsection{Traceability matrix}

Hệ thống đảm bảo traceability xuyên suốt từ Use Case → Diagram → Schema → UI:

\begin{table}[h]
\centering
\caption{Traceability Matrix ví dụ}
\label{tab:traceability}
\begin{tabular}{|l|l|l|l|}
\hline
\textbf{Use Case} & \textbf{Diagram} & \textbf{Schema} & \textbf{UI Screen} \\
\hline
UC01 Register & sequence-m1-a1.md & class-m1.yaml & SC-M1-01 \\
\hline
UC02 Login & sequence-m1-a2.md & class-m1.yaml & SC-M1-02 \\
\hline
UC03 Edit Profile & flow-m1-b3.md & class-m1.yaml & SC-M1-03 \\
\hline
\end{tabular}
\end{table}

% -----------------------------------------------------------------------------
\subsection{Knowledge Factory model}

Thiết kế này không chỉ tạo ra một website mạng xã hội cụ thể, mà quan trọng hơn
là thiết lập được một \textbf{"Knowledge Factory"} — nhà máy sản xuất tri thức
cho phép AI tự động hóa quy trình phát triển phần mềm một cách tin cậy và có kiểm soát.

Ưu điểm:
\begin{itemize}
    \item \textbf{Tái sử dụng}: Mỗi skill có thể áp dụng cho nhiều dự án khác
    \item \textbf{Nhất quán}: 3 Pillars + 7 Zones đảm bảo tính đồng nhất
    \item \textbf{Kiểm soát}: Interaction Points + Self-Scoring chống hallucination
    \item \textbf{Mở rộng}: Dễ dàng thêm skill mới theo framework chuẩn
\end{itemize}

% =============================================================================
% KẾT THÚC CHƯƠNG 2
% =============================================================================
